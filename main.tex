\documentclass[ngerman]{report}
\usepackage{graphicx}

\usepackage[main=ngerman]{babel}

\title{SWPP-Zusammenfassung}
    \author{Ella Köll, Elmira Parfjonova, Finn Nothdurfter}
    \begin{document}
    \maketitle
    \tableofcontents



    \chapter{Übung 1: GitHub Projekt erstellen}



    \chapter{Übung 2: Teamaufgabe Teil 1 \texttt{=>} Projektdefinition}
    

Das Projekt besteht in der Entwicklung einer App, die basierend auf aktuellen Wetterdaten passende Empfehlungen für tägliche Aktivitäten und Kleidung liefert. Es ist zeitlich begrenzt (bis Ende des Schuljahres) und einmalig, da es eine innovative Lösung darstellt. Der Umfang ist komplex, da die App verschiedene Datenquellen integrieren muss. Das Projekt arbeitet mit begrenzten finanziellen und materiellen Ressourcen. Eine klare Zieldefinition ist gegeben: Die App soll benutzerfreundlich und funktional sein.
Wir bilden eine Matrix Projektorganisation, da wir uns neben dem Projekt auch um sonstige schulische Pflichten kümmern müssen.

Ziele (Siehe Terminende oben):
*   funktionierende App
*   pünktliche Abgabe
*   positive Note
*   praktisches Interface / hohe Usability
*   automatische Aktualisierung

Personen in der Gruppe:
Projektleiter: Finn Nothdurfter
Stellvertretende Projektleitung: Ella Köll
Teilprojektleitung: Elmira Parfjonova
Projektmitarbeiter: Finn Nothdurfter, Ella Köll, Elmira Parfjonova
Auftraggeber: Professor Landerer + Professor Netzer
Anwender/Kunde: Professor Landerer + Professor Netzer


    \chapter{Übung 3: Teamaufgabe Teil 2 \texttt{=>} Projektstrukturplan, Projektablaufplan}



    \chapter{Übung 4: Teamaufgabe Teil 3 \texttt{=>} Projektumefeldanalyse, Risikoanalyse}
    \chapter{Übung 5: Teamaufgabe Teil 4 \texttt{=>} Kanban Board in GitHub + Userstorys zum Productbacklog hizufügen}
    \chapter{Übung 6: Latex Dokument erstellen}
    

    Abbildung \ref{fig:portrait} zeigt diesen Herrn.
    Die Details sind in Abbildung \ref{tab:teilnehmer}
    
    \begin{figure}[h]
        \centering
       
        \caption{Finn Gaygners}
        \label{fig:portrait}
    \end{figure}
    
    \begin{table}[h]
        \begin{center}
            \begin{tabular}{ l  l | c }
             Vorname & Nachname & Ort \\
             \hline
             Max & Mustermann & Imst \\  
             Susi & Sorglos & Tarrenz    
            \end{tabular}
        \end{center}
        \caption{Die Teilnehmer}
        \label{tab:teilnehmer}
    \end{table}
 
\end{document}